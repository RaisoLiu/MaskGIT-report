\clearpage
\section{Implementation Details}

fake_image = torch.randn(1, 3, 64, 64).to(device)
codebook_mapping, codebook_indices = model.encode_to_z(fake_image)
Encoder 輸出維度分析:
codebook_mapping 維度: torch.Size([1, 256, 16, 16])
codebook_indices 維度: torch.Size([1, 256])


在實作中,這次的任務是要了解 MultiHeadAttention 的運作方式,


說明 VQGAN 的工作流程?

說明 VQGAN 與 VQVAE 在這邊所扮演的角色?


\begin{algorithm}[H]
\caption{計算 Query 和 Key 之間的歐氏距離}
\begin{algorithmic}[1]
\REQUIRE query $\in \mathbb{R}^{B \times S \times D}$, key $\in \mathbb{R}^{B \times S \times D}$
\ENSURE dist $\in \mathbb{R}^{B \times S \times S}$
\STATE $q_{square} \gets \sum_{d=1}^D query_{:,:,d}^2$ \COMMENT{Shape: $[B,S,1]$}
\STATE $k_{square} \gets \sum_{d=1}^D key_{:,:,d}^2$ \COMMENT{Shape: $[B,S,1]$}
\STATE $qk \gets query \times key^T$ \COMMENT{Shape: $[B,S,S]$}
\STATE $dist \gets q_{square} + k_{square}^T - 2 \times qk$
\RETURN dist
\end{algorithmic}
\end{algorithm}





\subsection{The details of your model}
This is a model detail section.
MultiHeadAttention

\subsection{The details of your stage2 training}
This is a stage2 training detail section.

\subsection{The details of your inference for inpainting task}
This is an inference detail section. 