\section{Introduction}

這次的作業是使用 Masked Generative Image Transformer 來進行圖像修復。會由 VQGAN 把每個圖片製作成多個 token 組成的序列,當作 Ground Truth。我們需要利用這個 Ground Truth 來訓練一個 Bi-directional 的 Transformer 模型,舉例來說,這次的圖片輸入是 64x64 的圖片,然後我們會把這個圖片切成 4x4 的區域,接下來把區域稱之為 Token ,然後每個 Token embedding 到 256 dimension 的空間中,所以每個圖片來說會有 16x16 個 Token。由於這個模型是 Bi-directional 的 Transformer,這樣的設計可以讓模型在生成圖像的時候,可以考慮到其他Token。