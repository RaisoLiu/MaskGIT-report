\section{Introduction}
\subsection{Background}
MaskGIT (Masked Generative Image Transformer) \cite{chang2022maskgit} is a novel approach in the field of image generation that combines the power of transformers \cite{vaswani2017attention} with masked modeling techniques. This report explores the implementation and evaluation of MaskGIT for image generation tasks.

\subsection{Problem Statement}
Traditional image generation methods often face challenges in terms of quality, diversity, and computational efficiency. We aim to investigate how MaskGIT addresses these limitations and its potential applications in real-world scenarios, particularly in the context of recent advances in visual-language models \cite{radford2021learning}.

\subsection{Objectives}
\begin{itemize}
    \item Implement MaskGIT architecture
    \item Evaluate its performance on standard datasets
    \item Compare with existing state-of-the-art methods
    \item Analyze computational efficiency and resource requirements
\end{itemize}

\section{Methodology}
\subsection{Architecture Overview}
The MaskGIT architecture consists of several key components:
\begin{itemize}
    \item Image Tokenization
    \item Transformer Encoder
    \item Masked Modeling Strategy
    \item Decoding Process
\end{itemize}

\subsection{Implementation Details}
\subsubsection{Data Preprocessing}
\begin{itemize}
    \item Dataset selection and preparation
    \item Image normalization
    \item Tokenization process
\end{itemize}

\subsubsection{Model Architecture}
\begin{itemize}
    \item Transformer configuration
    \item Masking strategy
    \item Training procedure
\end{itemize}

\subsection{Training Process}
\begin{itemize}
    \item Loss functions
    \item Optimization techniques
    \item Training hyperparameters
\end{itemize}

\section{Results}
\subsection{Quantitative Evaluation}
\begin{itemize}
    \item FID scores
    \item Inception scores
    \item Perceptual similarity metrics
\end{itemize}

\subsection{Qualitative Analysis}
\begin{itemize}
    \item Generated image samples
    \item Visual quality assessment
    \item Diversity analysis
\end{itemize}

\subsection{Performance Comparison}
\begin{itemize}
    \item Comparison with baseline models
    \item Computational efficiency metrics
    \item Resource utilization analysis
\end{itemize}

\section{Discussion}
\subsection{Key Findings}
\begin{itemize}
    \item Model strengths and limitations
    \item Performance insights
    \item Implementation challenges
\end{itemize}

\subsection{Limitations}
\begin{itemize}
    \item Computational constraints
    \item Quality limitations
    \item Potential improvements
\end{itemize}

\subsection{Future Work}
\begin{itemize}
    \item Architecture improvements
    \item Application extensions
    \item Optimization opportunities
\end{itemize}

\section{Conclusion}
\subsection{Summary}
A comprehensive summary of the project findings and their implications.

\subsection{Contributions}
Key contributions to the field of image generation.

\subsection{Implications}
Practical implications and potential applications.
